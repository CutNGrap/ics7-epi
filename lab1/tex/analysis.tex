% !TeX spellcheck = ru_RU
%\chapter{Аналитический раздел}

\chapter{Ход работы}

\section{Цель работы}
Целью лабораторной работы №1 является освоение возможностей программы Microsoft Project для планирования проекта по разработке программного обеспечения.

Содержание проекта: Команда разработчиков из 16 человек занимается созданием карты города на основе собственного модуля отображения. Проект должен быть завершен в течение 6 месяцев. Бюджет проекта: 50 000 рублей.

\section{Тренировочное задание (вариант 1)}
При выполнении задания использовались следующие параметры среды:
\begin{itemize}
\item	Время работы с 8:00 до 17:00; 8-часовой рабочий день; 40 часов в неделю;
\item	Длительность измерялась в днях, трудозатраты в часах.
\item	Стоит настройка фиксированных ресурсов.
\end{itemize}
Дата начала работ: 01.03.2024
Дата завершения работ: 02.05.2024
На рисунке 1 представлена диаграмма Ганта.

\img{0.5}{1}{Диаграмма Ганта}

\section{Основное задание}
\subsection{Задание 1}
На рисунке 2 представлено заполнение длительности работы, объема работы и типа работы. Также установка количества рабочих часов в день и неделю. Помимо этого, указаны начала рабочей недели в понедельник и финансового года — в январе. Установлена продолжительность рабочего дня.

\img{0.6}{2}{Окно настроек расписания}

На рисунке 3 представлено заполнение выходных и праздничных дней. Празднечные дни заполнены с марта по октябрь. Это 8 марта, 3 мая, 10 мая, и 14 июня.

\img{0.5}{3}{Окно изменения рабочего времени}

\subsection{Задание 2}
На рисунке 4 представлено заполнение списка задач в соответствии с таблицей. Даты заполнены автоматически.


\img{0.5}{4}{Заполнение списка задач}

\subsection{Задание 3}
На рисунке 5 представлена группировка задач в соответствии с заданием.  У суммарных задач изменилась длительность — теперь она показывает наибольшую длительность из всех подзадач, так как предлагается, что задачи начитаюся в одно время и время выполнения фазы   определеняется временем наиболее длинной задачи.


\img{0.47}{5}{Группировка задач}

\subsection{Задание 4}
На рисунке 6 представлено заполнение связей. После этого действия изменилась длительность суммарных задач, так как была установлена последовательность, в которой эти задачи должны выполнятся.


\img{0.47}{6}{Заполнение связей}

\section{Выводы}
В ходе работы были освоены возможности программы Microsoft Project для планирования проекта по разработке программного обеспечения. Был  создан план проекта создания карты города. Была получена дата завершения работ — 18.09.2024. В итоге длительность проекта превысила 6 месяцев, обозначенных в задании, на 18 дней.
